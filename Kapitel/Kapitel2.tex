\chapter{Hintergrund}

Dieses Kapitel legt den Grundstein für diese Arbeit. Es liefert den theoretischen Hintergrund, sowie die Motivation für die Forschung zur Digitalisierung im Handwerk. Zuerst wird der Stand der Digitalisierung in Handwerksbetrieben, sowie kleinen und mittleren Unternehmen dargelegt, welcher Anreiz gegeben hat, dieses Thema genauer zu beleuchten. Anschließend wird herausgearbeitet, wie digitale Anwendungen bereits eingesetzt werden, um bei bestimmten Arbeitsschritten zu unterstützen. Dabei werden Spezifikationserstellung und Dienstleistungsunterstützung genauer betrachtet und mit Beispielen veranschaulicht. Zuletzt wird auf das Potenzial von Augmented Reality zur Visualisierung eingegangen und dies wiederum durch Beispielanwendungen veranschaulicht.

\section{Digitalisierung in Handwerksbetrieben, kleinen und mittleren Unternehmen}

Die Digitalisierung hält Einzug in den meisten Branchen, was in dieser modernen Zeit, in welcher auch im privaten Leben mehr und mehr Technik zum Einsatz kommt, nicht verwunderlich ist. Mithilfe neuer Technologien und Programmen lassen sich beispielsweise Abläufe beschleunigen, Genauigkeit und Effizienz verbessern, sowie die Erreichbarkeit steigern. Oft werden diese Technologien zum bewältigen des Verwaltungsaufwands genutzt, was kleinen und mittleren Unternehmen (\textbf{KMUs}), als auch Handwerksbetrieben mehr Zeit für ihre eigentliche Aufgabe lässt: das Handwerk \cite{noauthor_neue_nodate}. Unternehmen wollen sich diese neuen Chancen zu nutze machen und Investieren dazu in Umrüstung und Umstrukturierung von Unternehmensabläufen, sowie in die Integration von neuen Programmen und technischen Hilfsmitteln in ihr Arbeitsumfeld \cite{TODO: Buchkapitel}. Große, kommerzielle Unternehmen fällt das leichter als KMUs und Handwerksbetrieben \cite{hateful_six_krcmar}.

KMUs spielen aber ein wichtige Rolle in unserer Gesellschaft und Wirtschaft, weshalb ihr langsamer Fortschritt im Bereich Digitalisierung auf keinen Fall leicht zu nehmen ist. In den USA beispielsweise stellen KMUs 95\% aller Unternehmen und sind verantwortlich für zwei Drittel aller neuen Arbeitsplätze. Zusätzlich werden 60\% des Exports von ihnen gestemmt \cite{allocca_innovation_2006}. Diese Argumente machen KMUs zu einem lohnenswerten und interessanten Forschungsgebiet, was sich in den letzten Jahren auch deutlich zeigte.

Fries et al. \cite{hateful_six_krcmar} trug durch ihre Forschung sechs Faktoren zusammen, welche dafür verantwortlich sind, dass KMUs ihren größeren Konkurrenten in der Digitalisierung hinterher hängen. Das liegt an: 

\begin{enumerate}
	\item einem empfundenen Ungleichgewicht zwischen Risiken und Möglichkeiten,
	\item mangelnder Vereinbarkeit mit der täglichen Arbeitsroutine,
	\item einer schwierigen Eingliederung in individuelle Geschäftsprozesse,
	\item komplexen Infrastrukturinvestitionen,
	\item nicht oder kaum vorhandenem IT Knowhow und
	\item hohen Installationskosten für die Inbetriebnahme der neuen Technologien.
\end{enumerate}

KMUs denken bei dem Einsatz von neuen Technologien vermehrt an Risiken und Kosten, als an Chancen, die daraus entstehen oder andere positive Aspekte. Sie haben limitierte Ressourcen an Kapital und Arbeitskräften und können sich daher keine falschen Entscheidungen oder Investitionen erlauben \cite{allocca_innovation_2006}. Mit dem Adaptieren und Einsatz von Innovationen geht immer ein hohes finanzielles Risiko einher. KMUs können, im Gegensatz zu größeren Firmen, diese Kosten meist nicht über mehrere Projekte verteilen, um die Belastung zu minimieren \cite{rothwell_small_1989}. \\
Alloca und Kessler \cite{allocca_innovation_2006} wiederum schreiben KMUs mehr Risikobereitschaft zu, da das Management bzw. die Geschäftsführung meist aus Entrepreneurs besteht, die den Einstieg in die Industrie durch Wagnisse schaffen wollen oder geschafft haben. Ein solches Unternehmen muss, aufgrund der knapperen Ressourcen, schneller und effizienter arbeiten, um konkurrenzfähig zu bleiben.

Des weiteren spielt es eine große Rolle, ob neue Technologien den Arbeitsalltag tatsächlich erleichtern und wie diese von den Mitarbeitern aufgenommen und in ihre Routine integriert werden. Dies kann besonders bei älteren Mitarbeitern, die \textit{"schon immer so gemacht"} haben schwierig werden. Die Arbeitsprozesse müssen umgebaut werden, um die neuen Geräte oder Programme zu unterstützen und ihre Potential zu nutzen. Da oft niemand vorhanden ist, der sich ausschließlich mit \textit{Research \& Development} beschäftigt, sind diese Entscheidungen schwer zu treffen \cite{rothwell_small_1989}. Nichtsdestotrotz haben KMUs aufgrund weniger Bürokratie das Potenzial, schneller Änderungen zu adaptieren (\cite{kessler_innovation_1996} und \cite{kessler_speeding_1999}, zitiert in \cite{allocca_innovation_2006}).

Unternehmerische Manager können oft die steigende Komplexität der Firma, durch deren Wachstum und Integration von neuen Technologien, schwer überblicken, worunter die Eingliederungen von Neuerungen in Geschäftsprozesse leidet \cite{rothwell_small_1989}. Dafür sind weniger standardisierte Prozesse und Guidlines, ein weniger systematischer Management Style und generell die geringere Erfahrung des Managements im Vergleich zu großen Unternehmen verantwortlich \cite{allocca_innovation_2006}. 

Das Einbinden der neuen Technologien in die bestehende Infrastruktur des Unternehmens stellt zusätzliche Risiken und Kosten dar. Der Arbeitsplatz war meist ursprünglich nicht dafür designed, um schnell und einfach neue Geräte oder Programme zu integrieren. Teilweise lässt auch der Ort an dem gearbeitet wird den Einsatz generell nicht oder nur bedingt zu. Auf einer Baustelle oder in einem Kellergewölbe beispielsweise kann keine permanente WLAN Verbindung sichergestellt werden \cite{TODO: Buchkapitel}. Dadurch wird der Einsatz von digitalen Medien erschwert.

Da KMUs und vor allem Handwerksbetriebe oft aus wenigen Mitarbeitern bestehen, ist die Wahrscheinlichkeit hoch, dass kein Technikspezialist an Board ist. Das macht es nötig, sich fremde Expertise einzuholen, um in die richtige Technologie für den Betrieb zu investieren. Jedoch bringt das zusätzliche Kosten und Zeitaufwand mit sich, welche oft nicht verfügbar sind \cite{rothwell_small_1989}. Des weiteren müssen die Mitarbeiter für die neue Technologie geschult werden, wodurch wieder Zeit verloren geht. Fehler, die durch Anschaffung falscher technischer Hilfsmittel, oder durch unsachgemäßen, ineffizienten Einsatz dieser entstehen, kosten unnötig Geld. Das kann sich ein Unternehmen mit begrenzten Ressourcen nicht leisten \cite{allocca_innovation_2006}.

Die Schulung der Mitarbeiter ist durchaus zeit- und kostenaufwändig. Aber auch die Anschaffung und Installation der neuen Technologie birgt eine finanzielle Hürde für die Unternehmen. Zusätzliche, versteckte Kosten welche durch Probleme mit Patenten oder Auflagen des Staates entstehen, stellen eine große Belastung bei der Integration dar \cite{rothwell_small_1989}. 

Trotz dieser Hindernden Faktoren für KMUs und Handwerksbetriebe bieten sich viele Möglichkeiten für diese. Ritchie und Brindley \cite{ritchie_ict_2005} sehen großes Potential für diese Unternehmen im Bereich Digitalisierung mit ihren kommerziellen Konkurrenten mitzuhalten, da sie \textit{flexibler, effizienter und adaptiver} sind, was Grundvoraussetzungen für die Integration von Neuerungen ist. \\
Das bereits Produkte zur Unterstützung für Handwerksbetriebe und KMUs existieren und genutzt werden, wird im Folgenden genauer dargestellt.

\section{Unterstützung der Spezifikationserstellung mit digitalen Anwendungen}



\section{Dienstleistungsunterstützung mit digitalen Anwendungen}

\section{Augmented Reality als Visualisierungswerkzeug}