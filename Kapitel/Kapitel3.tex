\chapter{Methoden}

\section{Fokusgruppe}

asdfasf

\section{Ethnographie: Teilnehmende Beobachtung}

asdfasdf

\section{Gestaltungsorientierte Forschung: Implementieren eines Prototypen}

asdfasdf

\section{Designen des Fragebogens}

Diese Arbeit zielt darauf ab, einen theoretischen, sowie einen praktischen Beitrag abzuleiten. Dazu wurden Anmerkungen der Testpersonen während und nach den Probandentests notiert. Diese Informationen können dazu genutzt werden, Ideen zu sammeln und die Applikation zu verbessern, allerdings lassen sie sich nicht gut vergleichen. Sie stellen den praktischen Beitrag dar. Für den theoretischen Teil benötigt man vergleichbare Werte und Größen. Dazu wurde ein Evaluationsbogen konzipiert, der folgende Fragen beantworten soll:

\begin{itemize}
	\item Wie ist der Bezug des Handwerkers zu technischen Geräten?
	\item Kann der Handwerker sich vorstellen die Technologie in Zukunft zu nutzen?
\end{itemize}

